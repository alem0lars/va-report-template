%!TEX root = ../main.tex


\chapter{Vulnerability Assessment}\label{ch:vulnass}

  \section{Methodology}\label{sec:methodology-utilized}

    \begin{itemize}
      \item \textbf{Reconnaissance:}
        The tester would attempt to gather as much information as possible about
        the selected network. Reconnaissance can take two forms i.e.~active and
        passive. A passive attack is always the best starting point as this
        would normally defeat intrusion detection systems and other forms of
        protection etc. afforded to the network. This would usually involve
        trying to discover publicly available information by utilizing a web
        browser and visiting newsgroups etc. An active form would be more
        intrusive and may show up in audit logs and may take the form of an
        attempted DNS zone transfer or a social engineering type of attack.
      \item \textbf{Enumeration:}
        The tester would use varied operating system fingerprinting tools to
        determine what hosts are alive on the network and more importantly what
        services and operating systems they are running. Research into these
        services would then be carried out to tailor the test to the discovered
        services.
      \item \textbf{Scanning:}
        By use of vulnerability scanners all discovered hosts would be tested
        for vulnerabilities. The result would then be analysed to determine if
        there any vulnerabilities that could be exploited to gain access to a
        target host on a network.
    \end{itemize}

  \section{Vulnerability Metrics}\label{sec:vulnerability-metrics}

    The \textbf{Common Vulnerability Scoring System} (CVSS) provides an open
    framework for communicating the characteristics and impacts of IT
    vulnerabilities. Its quantitative model ensures repeatable accurate
    measurement while enabling users to see the underlying vulnerability
    characteristics that were used to generate the scores. Thus, CVSS is
    well suited as a standard measurement system for industries,
    organizations, and governments that need accurate and consistent
    vulnerability impact scores.

    CVSS is composed of three metric groups: Base, Temporal, and
    Environmental, each consisting of a set of metrics.

    These metric groups are described as follows:

    \begin{itemize}
      \item \textbf{Base:}
        Represents the intrinsic and fundamental characteristics of a
        vulnerability that are constant over time and user environments
      \item \textbf{Temporal:}
        Represents the characteristics of a vulnerability that change
        over time but not among user environments
      \item \textbf{Environmental:}
        Represents the characteristics of a vulnerability that
        are relevant and unique to a particular user's environment
    \end{itemize}

    The purpose of the CVSS base group is to define and communicate the
    fundamental characteristics of a vulnerability. This objective approach
    to characterizing vulnerabilities provides users with a clear and
    intuitive representation of a vulnerability. Users can then invoke the
    temporal and environmental groups to provide contextual information that
    more accurately reflects the risk to their unique environment. This
    allows them to make more informed decisions when trying to mitigate
    risks posed by the vulnerabilities.

    \subsection{CVSS v3.0 Ratings}\label{subsec:cvss-v3.0-ratings}

      \begin{center}
        \begin{tabularx}{0.4\textwidth}{c c}
          \toprule
          \textbf{Severity} & \textbf{Base Score Range} \\ \midrule
          Low    & 0.0 -- 3.9  \\ \midrule
          Medium & 4.0 -- 6.9  \\ \midrule
          High   & 7.0 -- 10.0 \\
          \bottomrule
        \end{tabularx}
      \end{center}

    \subsection{Metric Groups}\label{subsec:metric-groups}

      \subsubsection{Base Metrics}\label{subsubsec:base-metrics}

        The base metric group captures the characteristics of a vulnerability
        that are constant with time and across user environments. The Access
        Vector, Access Complexity, and Authentication metrics capture how the
        vulnerability is accessed and whether or not extra conditions are
        required to exploit it. The three impact metrics measure how a
        vulnerability, if exploited, will directly affect an IT asset, where the
        impacts are independently defined as the degree of loss of
        confidentiality, integrity, and availability. For example, a
        vulnerability could cause a partial loss of integrity and availability,
        but no loss of confidentiality.

        \paragraph{Access Vector (AV)}\label{par:access-vector-av}

          This metric reflects how the vulnerability is exploited. The possible
          values for this metric are listed below. The more remote an
          attacker can be to attack a host, the greater the vulnerability score.

          \begin{itemize}
            \item
              \textbf{Local (L)}: A vulnerability exploitable with only local
              access requires the attacker to have either physical access to the
              vulnerable system or a local (shell) account. Examples of locally
              exploitable vulnerabilities are peripheral attacks such as
              Firewire/USB DMA attacks, and local privilege escalations
              (e.g. sudo)
            \item
              \textbf{Adjacent Network (A)}: A vulnerability exploitable with
              adjacent network access requires the attacker to have access to
              either the broadcast or collision domain of the vulnerable
              software. Examples of local networks include local IP subnet,
              Bluetooth, IEEE 802.11, and local Ethernet segment
            \item
              \textbf{Network (N)}: A vulnerability exploitable with network
              access means the vulnerable software is bound to the network stack
              and the attacker does not require local network access or local
              access. Such a vulnerability is often termed ``remotely
              exploitable''. An example of a network attack is an RPC buffer
              overflow
          \end{itemize}

        \paragraph{Access Complexity (AC)}\label{par:access-complexity-ac}

          This metric measures the complexity of the attack required to exploit
          the vulnerability once an attacker has gained access to the target
          system. For example, consider a buffer overflow in an Internet service:
          once the target system is located, the attacker can launch an exploit at
          will.

          Other vulnerabilities, however, may require additional steps in order to
          be exploited. For example, a vulnerability in an email client is only
          exploited after the user downloads and opens a tainted attachment. The
          lower the required complexity, the higher the vulnerability score.

          \begin{itemize}
            \item
              \textbf{High (H)} Specialized access conditions exist. For example:

              \begin{itemize}
                \item
                  In most configurations, the attacking party must already have
                  elevated privileges or spoof additional systems in addition to the
                  attacking system (e.g., DNS hijacking).
                \item
                  The attack depends on social engineering methods that would be
                  easily detected by knowledgeable people. For example, the victim
                  must perform several suspicious or atypical actions.
                \item
                  The vulnerable configuration is seen very rarely in practice.
                \item
                  If a race condition exists, the window is very narrow.
              \end{itemize}
            \item
              \textbf{Medium (M)} The access conditions are somewhat specialized;
              the following are examples:

              \begin{itemize}
                \item
                  The attacking party is limited to a group of systems or users at
                  some level of authorization, possibly untrusted.
                \item
                  Some information must be gathered before a successful attack can be
                  launched.
                \item
                  The affected configuration is non-default, and is not commonly
                  configured (e.g., a vulnerability present when a server performs
                  user account authentication via a specific scheme, but not present
                  for another authentication scheme).
                \item
                  The attack requires a small amount of social engineering that might
                  occasionally fool cautious users (e.g., phishing attacks that modify
                  a web browsers status bar to show a false link, having to be on
                  someones buddy list before sending an IM exploit).
              \end{itemize}
            \item
              \textbf{Low (L)} Specialized access conditions or extenuating
              circumstances do not exist. The following are examples:

              \begin{itemize}
                \item
                  The affected product typically requires access to a wide range of
                  systems and users, possibly anonymous and untrusted (e.g.,
                  Internet-facing web or mail server).
                \item
                  The affected configuration is default or ubiquitous.
                \item
                  The attack can be performed manually and requires little skill or
                  additional information gathering.
                \item
                  The race condition is a lazy one (i.e., it is technically a race but
                  easily winnable).
              \end{itemize}
          \end{itemize}

          This metric measures the number of times an attacker must authenticate
          to a target in order to exploit a vulnerability. This metric does not
          gauge the strength or complexity of the authentication process, only
          that an attacker is required to provide credentials before an exploit
          may occur. The fewer authentication instances that are required, the
          higher the vulnerability score.

          \begin{itemize}
            \item
              \textbf{Multiple (M)} Exploiting the vulnerability requires that the
              attacker authenticate two or more times, even if the same credentials
              are used each time. An example is an attacker authenticating to an
              operating system in addition to providing credentials to access an
              application hosted on that system.
            \item
              \textbf{Single (S)} The vulnerability requires an attacker to be
              logged into the system (such as at a command line or via a desktop
              session or web interface).
            \item
              \textbf{None (N)} Authentication is not required to exploit the
              vulnerability.
          \end{itemize}

          The metric should be applied based on the authentication the attacker
          requires before launching an attack. For example, if a mail server is
          vulnerable to a command that can be issued before a user authenticates,
          the metric should be scored as ``None'' because the attacker can launch
          the exploit before credentials are required. If the vulnerable command
          is only available after successful authentication, then the
          vulnerability should be scored as ``Single'' or ``Multiple,'' depending
          on how many instances of authentication must occur before issuing the
          command.

        \paragraph{Confidentiality Impact (C)}\label{par:confidentiality-impact-c}

          This metric measures the impact on confidentiality of a successfully
          exploited vulnerability. Confidentiality refers to limiting information
          access and disclosure to only authorized users, as well as preventing
          access by, or disclosure to, unauthorized ones. Increased
          confidentiality impact increases the vulnerability score.

          \begin{itemize}
            \item
              \textbf{None (N)} There is no impact to the confidentiality of the
              system.
            \item
              \textbf{Partial (P)} There is considerable informational disclosure.
              Access to some system files is possible, but the attacker does not
              have control over what is obtained, or the scope of the loss is
              constrained. An example is a vulnerability that divulges only certain
              tables in a database.
            \item
              \textbf{Complete (C)} There is total information disclosure, resulting
              in all system files being revealed. The attacker is able to read all
              of the system's data (memory, files, etc.)
          \end{itemize}

        \paragraph{Integrity Impact (I)}\label{par:integrity-impact-i}

          This metric measures the impact to integrity of a successfully exploited
          vulnerability. Integrity refers to the trustworthiness and guaranteed
          veracity of information. Increased integrity impact increases the
          vulnerability score.

          \begin{itemize}
            \item
              \textbf{None (N)} There is no impact to the integrity of the system.
            \item
              \textbf{Partial (P)} Modification of some system files or information
              is possible, but the attacker does not have control over what can be
              modified, or the scope of what the attacker can affect is limited. For
              example, system or application files may be overwritten or modified,
              but either the attacker has no control over which files are affected
              or the attacker can modify files within only a limited context or
              scope.
            \item
              \textbf{Complete (C)} There is a total compromise of system integrity.
              There is a complete loss of system protection, resulting in the entire
              system being compromised. The attacker is able to modify any files on
              the target system.
          \end{itemize}

        \paragraph{Availability Impact (A)}\label{par:availability-impact-a}

          This metric measures the impact to availability of a successfully
          exploited vulnerability. Availability refers to the accessibility of
          information resources. Attacks that consume network bandwidth,
          processor cycles, or disk space all impact the availability of a
          system. Increased availability impact increases the vulnerability
          score.

          \begin{itemize}
            \item
              \textbf{None (N)} There is no impact to the availability of the
              system.
            \item
              \textbf{Partial (P)} There is reduced performance or interruptions
              in resource availability. An example is a network-based flood
              attack that permits a limited number of successful connections to
              an Internet service.
            \item
              \textbf{Complete (C)} There is a total shutdown of the affected
              resource. The attacker can render the resource completely
              unavailable.
          \end{itemize}

    \subsection{Temporal Metrics}\label{subsec:temporal-metrics}

      The threat posed by a vulnerability may change over time. Three such
      factors that CVSS captures are: confirmation of the technical details of
      a vulnerability, the remediation status of the vulnerability, and the
      availability of exploit code or techniques. Since temporal metrics are
      optional they each include a metric value that has no effect on the
      score. This value is used when the user feels the particular metric does
      not apply and wishes to ``skip over'' it.

      \subsubsection{Exploitability (E)}\label{subsubsec:exploitability-e}

        This metric measures the current state of exploit techniques or code
        availability. Public availability of easy-to-use exploit code increases
        the number of potential attackers by including those who are unskilled,
        thereby increasing the severity of the vulnerability.

        Initially, real-world exploitation may only be theoretical. Publication
        of proof of concept code, functional exploit code, or sufficient
        technical details necessary to exploit the vulnerability may follow.
        Furthermore, the exploit code available may progress from a
        proof-of-concept demonstration to exploit code that is successful in
        exploiting the vulnerability consistently. In severe cases, it may be
        delivered as the payload of a network-based worm or virus. The more
        easily a vulnerability can be exploited, the higher the vulnerability
        score.

        \begin{itemize}
          \item
            \textbf{Unproven (U)} No exploit code is available, or an exploit is
            entirely theoretical.
          \item
            \textbf{Proof-of-Concept (POC)} Proof-of-concept exploit code or an
            attack demonstration that is not practical for most systems is
            available. The code or technique is not functional in all situations
            and may require substantial modification by a skilled attacker.
          \item
            \textbf{Functional (F)} Functional exploit code is available. The code
            works in most situations where the vulnerability exists.
          \item
            \textbf{High (H)} Either the vulnerability is exploitable by
            functional mobile autonomous code, or no exploit is required (manual
            trigger) and details are widely available. The code works in every
            situation, or is actively being delivered via a mobile autonomous
            agent (such as a worm or virus).
          \item
            \textbf{Not Defined (ND)} Assigning this value to the metric will not
            influence the score. It is a signal to the equation to skip this
            metric.
        \end{itemize}

      \subsubsection{Remediation Level (RL)}\label{subsubsec:remediation-level-rl}

        The remediation level of a vulnerability is an important factor for
        prioritization. The typical vulnerability is unpatched when initially
        published. Workarounds or hotfixes may offer interim remediation until
        an official patch or upgrade is issued. Each of these respective stages
        adjusts the temporal score downwards, reflecting the decreasing urgency
        as remediation becomes final.The less official and permanent a fix, the
        higher the vulnerability score is.

        \begin{itemize}
          \item
            \textbf{Official Fix (OF)} A complete vendor solution is available.
            Either the vendor has issued an official patch, or an upgrade is
            available.
          \item
            \textbf{Temporary Fix (TF)} There is an official but temporary fix
            available. This includes instances where the vendor issues a temporary
            hotfix, tool, or workaround.
          \item
            \textbf{Workaround (W)} There is an unofficial, non-vendor solution
            available. In some cases, users of the affected technology will create
            a patch of their own or provide steps to work around or otherwise
            mitigate the vulnerability.
          \item
            \textbf{Unavailable (U There)} is either no solution available or it is
            impossible to apply.
          \item
            \textbf{Not Defined (ND)} Assigning this value to the metric will not
            influence the score. It is a signal to the equation to skip this
            metric.
        \end{itemize}

      \subsubsection{Report Confidence (RC)}\label{subsec:report-confidence-rc}

        This metric measures the degree of confidence in the existence of the
        vulnerability and the credibility of the known technical details.
        Sometimes, only the existence of vulnerabilities are publicized, but
        without specific details. The vulnerability may later be corroborated
        and then confirmed through acknowledgement by the author or vendor of
        the affected technology. The urgency of a vulnerability is higher when a
        vulnerability is known to exist with certainty. This metric also
        suggests the level of technical knowledge available to would-be
        attackers. The possible values for this metric are listed below.
        The more a vulnerability is validated by the vendor or other reputable
        sources, the higher the score.

        \begin{itemize}
          \item
            \textbf{Unconfirmed (UC)} There is a single unconfirmed source or
            possibly multiple conflicting reports. There is little confidence in
            the validity of the reports. An example is a rumor that surfaces from
            the hacker underground.
          \item
            \textbf{Uncorroborated (UR)} There are multiple non-official sources,
            possibly including independent security companies or research
            organizations. At this point there may be conflicting technical
            details or some other lingering ambiguity.
          \item
            \textbf{Confirmed (C)} The vulnerability has been acknowledged by the
            vendor or author of the affected technology. The vulnerability may
            also be Confirmed when its existence is confirmed from an external
            event such as publication of functional or proof-of-concept exploit
            code or widespread exploitation.
          \item
            \textbf{Not Defined (ND)} Assigning this value to the metric will not
            influence the score. It is a signal to the equation to skip this
            metric.
        \end{itemize}

    \subsection{Environmental Metrics}\label{subsec:environmental-metrics}

      Different environments can have an immense bearing on the risk that a
      vulnerability poses to an organization and its stakeholders. The CVSS
      environmental metric group captures the characteristics of a
      vulnerability that are associated with a user's IT environment. Since
      environmental metrics are optional they each include a metric value that
      has no effect on the score. This value is used when the user feels the
      particular metric does not apply and wishes to ``skip over'' it.

      \subsubsection{Collateral Damage Potential (CDP)}\label{subsubsec:collateral-damage-potential-cdp}

        This metric measures the potential for loss of life or physical assets
        through damage or theft of property or equipment. The metric may also
        measure economic loss of productivity or revenue. The possible values
        for this metric are listed below. Naturally, the greater the
        damage potential, the higher the vulnerability score.

        \begin{itemize}
          \item
            \textbf{None (N)} There is no potential for loss of life, physical
            assets, productivity or revenue.
          \item
            \textbf{Low (L)} A successful exploit of this vulnerability may result
            in slight physical or property damage. Or, there may be a slight loss
            of revenue or productivity to the organization.
          \item
            \textbf{Low-Medium (LM)} A successful exploit of this vulnerability
            may result in moderate physical or property damage. Or, there may be a
            moderate loss of revenue or productivity to the organization.
          \item
            \textbf{Medium-High (MH)} A successful exploit of this vulnerability
            may result in significant physical or property damage or loss. Or,
            there may be a significant loss of revenue or productivity.
          \item
            \textbf{High (H)} A successful exploit of this vulnerability may
            result in catastrophic physical or property damage and loss. Or, there
            may be a catastrophic loss of revenue or productivity.
          \item
            \textbf{Not Defined (ND)} Assigning this value to the metric will not
            influence the score. It is a signal to the equation to skip this
            metric.
        \end{itemize}

        Clearly, each organization must determine for themselves the precise
        meaning of ``slight, moderate, significant, and catastrophic.''

      \subsubsection{Target Distribution (TD)}\label{subsubsec:target-distribution-td}

        This metric measures the proportion of vulnerable systems. It is meant
        as an environment-specific indicator in order to approximate the
        percentage of systems that could be affected by the vulnerability. The
        possible values for this metric are listed below. The greater the
        proportion of vulnerable systems, the higher the score.

        \begin{itemize}
          \item
            \textbf{None (N)} No target systems exist, or targets are so highly
            specialized that they only exist in a laboratory setting. Effectively
            0\% of the environment is at risk.
          \item
            \textbf{Low (L)} Targets exist inside the environment, but on a small
            scale. Between 1\% - 25\% of the total environment is at risk.
          \item
            \textbf{Medium (M)} Targets exist inside the environment, but on a
            medium scale. Between 26\% - 75\% of the total environment is at risk.
          \item
            \textbf{High (H)} Targets exist inside the environment on a
            considerable scale. Between 76\% - 100\% of the total environment is
            considered at risk.
          \item
            \textbf{Not Defined (ND)} Assigning this value to the metric will not
            influence the score. It is a signal to the equation to skip this
            metric.
        \end{itemize}

      \subsubsection{Security Requirements (CR, IR, AR)}\label{subsubsec:security-requirements-cr-ir-ar}

        These metrics enable the analyst to customize the CVSS score depending
        on the importance of the affected IT asset to a users organization,
        measured in terms of confidentiality, integrity, and availability, That
        is, if an IT asset supports a business function for which availability
        is most important, the analyst can assign a greater value to
        availability, relative to confidentiality and integrity. Each security
        requirement has three possible values: low, medium, or high.

        The full effect on the environmental score is determined by the
        corresponding base impact metrics (please note that the base
        confidentiality, integrity and availability impact metrics, themselves,
        are not changed). That is, these metrics modify the environmental score
        by reweighting the (base) confidentiality, integrity, and availability
        impact metrics. For example, the confidentiality impact (C) metric has
        increased weight if the confidentiality requirement (CR) is high.
        Likewise, the confidentiality impact metric has decreased weight if the
        confidentiality requirement is low. The confidentiality impact metric
        weighting is neutral if the confidentiality requirement is medium. This
        same logic is applied to the integrity and availability requirements.

        Note that the confidentiality requirement will not affect the
        environmental score if the (base) confidentiality impact is set to none.
        Also, increasing the confidentiality requirement from medium to high
        will not change the environmental score when the (base) impact metrics
        are set to complete. This is because the impact sub score (part of the
        base score that calculates impact) is already at a maximum value of 10.

        The possible values for the security requirements are listed below. For
        brevity, the same table is used for all three metrics. The greater the
        security requirement, the higher the score (remember that medium is
        considered the default). These metrics will modify the score as much as
        plus or minus 2.5.

        \begin{itemize}
          \item
            \textbf{Low (L)} Loss of {[}confidentiality / integrity /
            availability{]} is likely to have only a limited adverse effect on the
            organization or individuals associated with the organization (e.g.,
            employees, customers).
          \item
            \textbf{Medium (M)} Loss of {[}confidentiality / integrity /
            availability{]} is likely to have a serious adverse effect on the
            organization or individuals associated with the organization (e.g.,
            employees, customers).
          \item
            \textbf{High (H)} Loss of {[}confidentiality / integrity /
            availability{]} is likely to have a catastrophic adverse effect on the
            organization or individuals associated with the organization (e.g.,
            employees, customers).
          \item
            \textbf{Not Defined (ND)} Assigning this value to the metric will not
            influence the score. It is a signal to the equation to skip this
            metric.
        \end{itemize}


<%# vim: set filetype=eruby.tex : %>
