%!TEX root = ../main.tex


\chapter{Introduction}\label{ch:introduction}

  <%# 1. Personnel involved in the testing from both the Client and Penetration
         Testing Team %>
  The following document was written as a reference for the security
  assessment done by <%= context[:authors].map do |author|
    author[:name]
  end.join(", ") %> towards <%= context[:customer][:name] %> on <%= date %>.

  <%# 2. Contact information %>
  <%# TODO %>

  <%# 3. Assets involved in testing %>
  \section{Assets involved}\label{sec:assets}

    The following assets were involved during the testing activities:

    \begin{itemize}
      <% context[:assets].each do |asset| %>
        <% if asset.key? :description %>
          \item \textbf{<%= asset[:name] %>}: <%= asset[:description] %>
        <% else %>
          \item \textbf{<%= asset[:name] %>}
        <% end %>
      <% end %>
    \end{itemize}

  <%# 4. Objectives of Test %>

    <%# TODO %>

  <%# 5. Scope of Test %>
  \section{Context \& Scope}\label{sec:scope}

    This audit has been carried out at the request of
    <%= context[:customer][:name] %>. Its goal was to evaluate the security of
    the assets under test, as described in section: \nameref{sec:assets}.

    The study focuses on main Web Vulnerabilities as described in OWASP Web
    Application Penetration Testing Methodology. In particular the following
    areas have been considered:

    \begin{itemize}
      \item Configuration and Deployment Management Testing
      \item Identity Management Testing
      \item Authentication Testing
      \item Authorization Testing
      \item Session Management Testing
      \item Input Validation Testing
      \item Error Handling Testing
      \item Cryptography Testing
      \item Business Logic Testing
      \item Client Side Testing
    \end{itemize}

  <%# 6. Strength of Test %>
  <%# TODO %>

  <%# 7. Approach %>
  <%# TODO %>

  <%# 8. Threat / Grading Structure %>
  <%# TODO %>


<%# vim: set filetype=eruby.tex : %>
