%!TEX root = ../main.tex


\chapter{Introduction}\label{cha:introduction}

  <%# 1. Personnel involved in the testing from both the Client and Penetration
         Testing Team %>
  The following document was written as a reference for the security
  assessment done by <%= context[:authors].map do |author|
    author[:name]
  end.join(", ") %> towards <%= context[:customer][:name] %> on <%= date %>.

  <%# 2. Contact information %>
  <%# TODO %>

  <%# 3. Assets involved in testing %>
  \section{Assets involved}\label{sec:assets-involved}

    The following assets were involved during the testing activities:

    \begin{itemize}
      <% context[:assets].each do |asset| %>
        <% if asset.key? :description %>
          \item \textbf{<%= asset[:name] %>}: <%= asset[:description] %>
        <% else %>
          \item \textbf{<%= asset[:name] %>}
        <% end %>
      <% end %>
    \end{itemize}

  <%# 4. Objectives of Test %>
  <%# TODO %>

  <%# 5. Scope of Test %>
  <%# TODO %>

  <%# 6. Strength of Test %>
  \section{Strength of Test}\label{sec:strength-test}

  <%# TODO if invasiveTestDone => Strength = High %>

  The tests were executed in Safe Check. Doing so there is no possibilities to
  disrupt the services during the vulnerability scans. However, this safer
  approach entails some risk and a lower level of strength. Not all exploit were
  tested nor DoS/DDoS attacks were performed. External attackers try to break
  into the system without worrying about what can be or can not be damaged.

  <%# 7. Approach %>
  \section{Approach}\label{sec:approach}

  <%# TODO if Black-Box %>

  The team has done a \textbf{Black-Box Vulnerability Assessment} and
  Penetration Test. No prior knowledge of a company network is known except the
  two IP mentioned above. In essence an example of this is when an external web
  based test is to be carried out and only the details of a website URL or IP
  address is supplied to the testing team. It would be their role to attempt to
  break into the company website/ network. This would equate to an external
  attack carried out by a malicious hacker.

  <%# 8. Threat / Grading Structure %>
  \section{Threat}\label{sec:threat}

  A threat is a potential for violation of security, which exists when there is
  a circumstance, capability, action, or event that could breach security and
  cause harm. In this particular scenario we try to anticipate and overcome
  problems from malicious agents (both human and software) intentional or
  unintentional.


  <%# vim: set filetype=eruby.tex : %>
